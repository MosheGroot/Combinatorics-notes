\documentclass{article}
\usepackage[T1,T2A]{fontenc}
\usepackage[utf8]{inputenc}
\usepackage[english,russian]{babel}

\usepackage[left=2cm,right=2cm,
    top=1cm,bottom=2cm,bindingoffset=0cm]{geometry}

\usepackage{graphicx}
\usepackage{color}
\usepackage{hyperref}
\usepackage{csquotes}

\usepackage{amsmath}
\usepackage{setspace}
\usepackage{indentfirst}
\usepackage{textcomp}
\usepackage{ifthen}
\usepackage{calc}

\title{Комбинаторика}
\author{Конспект по книге "Комбинаторика - Я. Н. Виленкин (и др.) 2006 г." }

\begin{document}
\maketitle
\tableofcontents

\section{Размещения, перестановки, сочетания}

\subsection{Размещения без повторений}

В прошлый раз мы рассматривали размещения с повторениями, возможное количество которых определялось формулой $\overline{A}_n^k$, где $n$ каждая из $k$ позиций могла быть одним из $n$ вариантов (опять же, с допускаемыми повторениями).

Размещения \textit{без повторений} подразумевают уникальность объектов в множестве с $n$ элементами, а потому после выбора одного из элементов $n$ в дальнейшем рассматривается выбор из $n-1$ элементов.

Таким образом, мы вводим понятие \textbf{размещения без повторений из $n$ элементов по $k$}, и количество таких размещений определяется следующей формулой:
$$A_n^k = \frac{n!}{(n-k)!}$$

Заметим, что о порядке в данном случае нет и речи.

\subsection{Пример. Выборы.}

Текст задачи:
\begin{displayquote}
В правление избрано $9$ человек. Из них надо выбрать председателя, заместителя и секретаря. Сколькими способами можно это сделать?
\end{displayquote}

В задаче у нас $n = 9$ уникальных элементов (людей), которых надо распределить по $k = 3$ уникальным местам. Итого, количество способов находится следующим образом: $A_9^3 = \frac{9!}{6!} = 9 * 8 * 7 = 504$ вариантов.

\end{document}
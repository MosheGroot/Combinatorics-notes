\documentclass{article}
\usepackage[T1,T2A]{fontenc}
\usepackage[utf8]{inputenc}
\usepackage[english,russian]{babel}

\usepackage[left=2cm,right=2cm,
    top=1cm,bottom=2cm,bindingoffset=0cm]{geometry}

\usepackage{graphicx}
\usepackage{color}
\usepackage{hyperref}
\usepackage{csquotes}

\usepackage{amsmath}
\usepackage{setspace}
\usepackage{indentfirst}
\usepackage{textcomp}
\usepackage{ifthen}
\usepackage{calc}

\title{Комбинаторика}
\author{Конспект по книге "Комбинаторика - Я. Н. Виленкин (и др.) 2006 г." }

\begin{document}
\maketitle
\tableofcontents

\section{Общие правила комбинаторики}

\subsection{Правило суммы}
Определение без использования множеств можно описать так:

\begin{displayquote}
Если некоторый объект $A$ можно выбрать $m$ способами, а другой объект $B$ можно выбрать $n$ способами, то выбор \textbf{"либо $A$, либо $B$"} можно осуществить $m + n$ способами.
\end{displayquote}

При том, при использовании этого правила нужно следить, чтобы ни один из способов выбора объекта $A$ не совпадал с способом выбора объекта $B$, иначе эти совпадения следует вычесть из общего результата: $m + n - k$, где $k$ -- количество совпадающих способов выбора.

\subsection{Правило произведения}

Определение:
\begin{displayquote}
Если объект $A$ можно выбрать $m$ способами, 
и если после каждого такого выбора объект $B$ можно выбрать $n$ способами, 
то выбор пары $(A; B)$ в указанном порядке можно осуществить $mn$ способами.
\end{displayquote}

Конечно, при наличии различных $a_1, a_2, a_3, ..., a_k$ элементов, 
которые можно выбрать последовательно способами $n_1, n_2, n_3, ..., n_k$,
и пользуясь математической индукцией, количество выборов последовательностей $(a_1; a_2; a_3; ...; a_k)$ 
можно рассчитать как $n_1n_2n_3...n_k$.

\subsection{Пример. Задача на домино.}

Текст задачи:
\begin{displayquote}
Сколькими способами из $28$ костей домино можно выбрать две кости так, 
чтобы их можно было приложить друг к другу (т.е. чтобы какое-то число встречалась на обоих костях)
\end{displayquote} 

Для решения попробуем сначала выбрать первую кость:
\begin{itemize}
    \item Мы можем выбрать один из $7$ дублей: $[0|0]$, $[1|1]$, $[2|2]$, $[3|3]$, $[4|4]$, $[5|5]$, $[6|6]$
    \item Мы можем выбрать любую из $21$ остальных костей с разными числами на концах ($[0|1]$, ..., $[5, 6]$)
\end{itemize}

Далее попробуем выбрать вторую кость к первой в зависимости от выбранного ранее варианта 
\begin{itemize}
	\item Если мы выбрали один из $7$ дублей вида $[X|X]$, 
	то к нему нам подойдет одна из $6$ костей вида $[X|0]$, ..., $[X|6]$ (кроме кости $[X|X]$, ведь её мы уже взяли)
	\item Если мы выбрали "обычную" кость вида $[X|Y]$, то к ней существует $6$ других костей с $X$ 
	($[X|0]$, ..., $[X|6]$, за исключением $[X|Y]$) и $6$ других костей с $Y$
	($[Y|0]$, ..., $[Y|6]$, за исключением $[X|Y]$), то есть всего 12 различных костей
\end{itemize}

Итого, по правилу произведения при выборе дубля в качестве первой кости получаем $7 * 6 = 42$ варианта, 
а во втором случае получаем $21 * 12 = 252$ варианта последовательного выбора двух костей. 
По правилу сложения: $42 + 252 = 294$.

Однако, учтем также, что порядок выбора не играет роли, то есть в нашем ответе варианты продублированы (взять $[0|0]$, а потом $[0|1]$ -- это то же самое, что взять $[0|1]$, а потом $[0|0]$), так что итоговое количество вариантов равняется $\frac{294}{2} = 147$.


\subsection{Размещения с повторением}

Грубо говоря, это определение частного случая, которое вытекает из правила произведения:
\begin{displayquote}
Упорядоченная выборка $k$ элементов с повторениями из выборки $n$ элементов называется \textbf{размещением с повторениями} из $n$ элементов по $k$ мест, и рассчитывается по формуле:

$$
\overline{A}_n^k = n^k
$$
\end{displayquote} 

Простейший пример использования: расчет количества допустимых вариантов для $k$-битного слова/числа (в данном случае будет $n = 2$, так как мы говорим о двоичной системе счисления)

\subsection{Пример. Задача на доставку писем.}

Текст задачи:
\begin{displayquote}
Требуется срочно доставить 6 писем разным адресатам. Сколькими способами это можно сделать, если можно послать трех курьеров, и каждое письмо можно дать любому из курьеров? (Порядок доставки не играет роли)
\end{displayquote} 

Решение максимально простое: $$\overline{A}_3^6 = 3^6 = 729$$ 

Важно только обратить внимание, что фактически требуется распределить курьеров по письмам, а не наоборот, так как мы вполне можем распределить одного курьера на все $6$ писем, а при распределении $\overline{A}_6^3$ мы бы фактически распределили по одному письму каждому из $3$ курьеров (более того, был бы реален случай, что двум курьерам дано одно письмо), что, очевидно, ошибочно.

\subsection{Формула включений и исключений}

Пусть у нас есть задача, в которой предметы могут обладать различными свойствами $a_1, a_2, ..., a_n$, при том предметов с одинаковым набором свойств может быть несколько. Количество таких предметов с одинаковыми свойствами будем обозначать как $N(a_ia_j...a_k)$, а отрицание какого-либо свойства будем обозначать черной над ним: $N(a_i\overline{a_j}...a_k)$.

\begin{displayquote}
Для нахождения количества всех предметов $N(\overline{a_1}\overline{a_2}...\overline{a_n})$, не удовлетворяющих ни одному свойству, применяется формула:

\begin{align*}
N(\overline{a_1}\overline{a_2}...\overline{a_n}) &= N \\
    &- N(a_1) - N(a_2) - ... - N(a_n) \\
    &+ N(a_1a_2) + N(a_1a_3) + ... + N(a_1a_n) + N(a_2a_3) + ... + N(a_{n-1}a_n)  \\
    &- N(a_1a_2a_3) - N(a_1a_2a_4) - ... - N(a_{n-2}a_{n-1}a_n)  \\
    &+ ... + (-1)^nN(a_1a_2...a_n)
\end{align*}

\end{displayquote}

То есть попеременно вычитаются количества предметов с четным количеством свойств и вычитаются с нечетным количеством свойств. Для $n = 3$ формула будет выглядеть следующим образом:
\begin{align*}
N(\overline{a_1}\overline{a_2}\overline{a_3}) &= N \\
    &- N(a_1) - N(a_2) - N(a_3) \\
    &+ N(a_1a_2) + N(a_1a_3) + N(a_2a_3) \\
    &- N(a_1a_2a_3) 
\end{align*}

Также формулу можно использовать интерпретировать для случая, когда необходимо исследовать предметы, обладающие конкретным свойством $a_n$:

\begin{align*}
N(\overline{a_1}\overline{a_2}...\overline{a_{n-1}}a_n) &= N(a_n) \\
    &- N(a_1a_n) - ... - N(a_1{n-1}a_n) \\
    &+ N(a_1a_2a_n)  - ... - N(a_{n-2}a_{n-1}a_n)  \\
    &+ ... + (-1)^{n-1}N(a_1a_2...a_{n-1}a_n)
\end{align*}

Та же формула при $n = 3$:
\begin{align*}
N(\overline{a_1}\overline{a_2}a_3) &= N(a_3) \\
    &- N(a_1a_3) - N(a_2a_3) \\
    &+ N(a_1a_2a_3) 
\end{align*}

\subsection{Пример. Решето Эратосфена.}

Данная задача использует идею решета Эратосфена, но мы используем его не для всех чисел. Текст задачи:
\begin{displayquote}
Сколько чисел от $0$ до $999$ не делятся ни на 5, ни на 7?
\end{displayquote} 

Обозначим за $a_x$ свойство деления на $x$, а $a_5$ и $a_7$ -- свойство деления на $5$ и на $7$ соответственно.

Тогда найдем количество чисел, делящихся на $5$, $7$ и $5 * 7 = 35$:
\begin{itemize}
	\item $N(a_5) = 1000 / 5 = 200$
	\item $N(a_7) = 1000 / 7 = 142$ (с округлением вниз)
	\item $N(a_5a_7) = \frac{1000}{5 * 7} = 28$ (с округлением вниз)
\end{itemize} 

Тогда итоговый результат $N(\overline{a_5a_7}) = 1000 - 200 - 142 + 28 = 686$.

\subsection{Пример. Знающие иностранные языки.}

\begin{displayquote}
В некотором отделе: 6 человек знают английский, 7 -- французский, 6 -- немецкий, 2 знают английский и французский, 4 -- английский и немецкий, 3 -- французский и немецкий, 1 человек знает все 3 языка.

Сколько человек работает в отделе? Сколько из них знают только английский язык?
\end{displayquote} 

Обозначим свойства буквами названия языка на английском: $a_e$, $a_f$, $a_g$ 

Общее количество человек рассчитаем по формуле: 
$$N = N(a_e) + N(a_f) + N(a_g) - N(a_ea_f) - N(a_ea_g) - N(a_fa_g) + N(a_ea_fa_g) = 6 + 7 + 6 - 2 - 4 - 3 + 1 = 11$$.

Знающих только английский язык (свойство $a_e$) рассчитаем по формуле: 
$$N(a_e\overline{a_fa_g}) = N(a_e) - N(a_ea_f) - N(a_ea_g) +  N(a_ea_fa_g) = 6 - 2 - 4 + 1 = 1$$

\end{document}